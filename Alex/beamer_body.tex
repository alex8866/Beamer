\input{beamer_head}

% 使用 \part,\section,\subsection 等命令组织文档结构
% 使用 \frame 命令制作幻灯片

\begin{document}

\logo{\includegraphics[height=0.09\textwidth]{redhat.jpg}}
\title[Beamer模版演示]{使用基于\XeTeX 的Beamer制作幻灯片的模版}
\author[Yixf]{Yi Xianfu}
\institute[IHS, SIBS, CAS]{Institute of Health Sciences\\ Shanghai Institutes for Biological Sciences\\ Chinese Academy of Sciences}
\date{\today}

% 定义目录页
\AtBeginPart{
  \frame{
    \frametitle{\partpage}
    \begin{multicols}{2}
% 如果目录过长,可以打开这个选项分两栏显示
      \tableofcontents
% 使用这个命令自动生成目录
    \end{multicols}
  }  
}

% 在每个Section前都会加入的Frame
\AtBeginSection[]
{
  \begin{frame}<beamer>
    \frametitle{提纲}
	\setcounter{tocdepth}{2}
    \tableofcontents[currentsection,currentsubsection]
  \end{frame}
}
% 在每个Subsection前都会加入的Frame
\AtBeginSubsection[]
{
  \begin{frame}<beamer>
%\begin{frame}<handout:0>
% handout:0 表示只在手稿中出现
    \frametitle{提纲}
	\setcounter{tocdepth}{2}
    \tableofcontents[currentsection,currentsubsection]
% 显示在目录中加亮的当前章节
  \end{frame}
}

\begin{frame}
  \titlepage
\end{frame}

\begin{frame}[plain]
  \frametitle{提纲}
  \setcounter{tocdepth}{2}
  \tableofcontents
\end{frame}

\section{列表}
\subsection{有序列表}
\begin{frame}
	\frametitle{有序列表}
	  \begin{enumerate}[<+-|alert@+>]
		  \item 有序列表一
		  \item 有序列表二
		  \item 有序列表三
		  \item 有序列表四
		  \item 有序列表五
		  \item 有序列表六
		  \item 有序列表七
		  \item 有序列表八
		  \item 有序列表九
	  \end{enumerate}
\end{frame}
\subsection{无序列表}
\begin{frame}
	\begin{itemize}
		\hilite <1> \item 无序列表一
		\hilite <2-3> \item 无序列表二
		\hilite <3> \item 无序列表三
		\hilite <4-> \item 无序列表四
		\hilite <5> \item 无序列表五
		\hilite <-6> \item 无序列表六
		\hilite <7> \item 无序列表七
		\hilite <8> \item 无序列表八
		\hilite <9> \item 无序列表九
	\end{itemize}
\end{frame}

\section{图片}
\begin{frame}
	\frametitle{图片}
	\begin{figure}[htbp]
    \centering
    \includegraphics[width=8cm]{power.png}
    \caption{Powered by}
    \label{fig:power}
    \end{figure}
\end{frame}

\section{表格}
\begin{frame}
	\frametitle{表格}
    \begin{table}
    \centering
    \caption{一个表格示例}
    \rowcolors[]{1}{blue!20}{blue!10}
    \begin{tabular}{|c|c|c|c|}
    \hline
    \rowcolor{blue!50}学号&姓名&年龄&成绩\\
    \hline
    001 & 丁一 & 25 & 91\\
    002 & 刘二 & 24 & 89\\
    003 & 张三 & 25 & 90\\
    004 & 李四 & 24 & 95\\
    005 & 王五 & 25 & 93\\
    006 & 赵六 & 26 & 88\\
    007 & 钱七 & 25 & 86\\
    008 & 王八 & 24 & 89\\
    009 & 孙九 & 25 & 90\\
    \hline
    \end{tabular}
    \end{table}
\end{frame}

\section{块状结构}
\subsection{内置}
\begin{frame}
	\frametitle{block, definition, example}
	\begin{block}{块}
	这是一个block。
	\end{block}
	\pause
	\begin{definition}{定义:}
	这是一个definition。
	\end{definition}
	\pause
	\begin{example}{实例:}
	这是一个example。
	\end{example}
\end{frame}
\subsection{自定义}
\begin{frame}
	\begin{beamercolorbox}[rounded=true,shadow=true,wd=12cm]{bgcolor}
		这是一个自定义的彩色块状结构。
	\end{beamercolorbox}
\end{frame}

\section{分栏}
\begin{frame}
	\frametitle{分栏}
	\begin{columns}
	\column{3cm}
	这是第一栏的文字;栏宽3cm。
	\column{5cm}
	这是第二栏的文字;栏宽5cm。
	\end{columns}
\end{frame}

\section{公式}
\begin{frame}
	\frametitle{质能方程}
	\begin{equation}
	E=mc^2
	\label{emc}
	\end{equation}
\end{frame}

\begin{frame}
\end{frame}

\section*{Acknowledgements}
\begin{frame}
	\frametitle{Powered by}
	\begin{center}
		\includegraphics[width=9cm]{power.png}
	\end{center}
\end{frame}

\begin{frame}[plain]
	\begin{spacing}{1.5}
	\begin{center}
	\Huge{\textbf{Thanks for your attention!}}

	\Huge{\textit{Any questions?}}
\end{center}
\end{spacing}
\end{frame}

\end{document}
