\documentclass[table]{beamer}
%[]中可以使用handout、trancompress等参数

%指定package
\usepackage{subfigure}
\usepackage{manfnt}%%% Dangerous Bend Symbols}\dbend \lhdbend \reversedvideodbend \textdbend \textlhdbend
\usepackage{multicol}
\usepackage{bbding}% 手势 \HandRight \HandLeft %\FiveStar \FourStar \SixStar
\usepackage{wasysym}
\usepackage{wasysym}
\usepackage{mflogo}

%指定beamer的模式与主题
\mode<presentation>
{
  \usetheme{CambridgeUS}
%Good themes
%\usetheme{CambridgeUS}
%\usetheme{Antibes}
%\usetheme{Madrid}
%\usetheme{Berkeley}
%\usetheme{Boadilla}

%\usecolortheme{default}
%\usecolortheme{orchid}
%\usecolortheme{whale}
\usecolortheme{spruce}
%\usefonttheme{professionalfonts}
}

%这里还可以选择别的主题:Bergen, Boadilla, Madrid, AnnArbor, CambridgeUS, Pittsburgh, Rochester, Warsaw, ...
%有导航栏的Antibes, JuanLesPins, Montpellier, ...
%有内容的Berkeley, PaloAlto, Goettingen, Marburg, Hannover, ...
%有最小导航栏的Berlin, Ilmenau, Dresden, Darmstadt, Frankfurt, Singapore, Szeged, ...
%有章和节表单的Copenhagen, Luebeck, Malmoe, Warsaw, ...

%\usecolortheme{default}
%设置内部颜色主题(这些主题一般改变block里的颜色);这个主题一般选择动物来命名
%这里还可以选择别的颜色主题,如默认的和有特别目的的颜色主题default,structure,sidebartab,全颜色主题albatross,beetle,crane,dove,fly,seagull,wolverine,beaver

%\usecolortheme{orchid}
%设置外部颜色主题(这些主题一般改变title里的颜色);这个主题一般选择植物来命名
%这里还可以选择别的颜色主题,如默认的和有特别目的的颜色主题lily,orchid,rose

%\usecolortheme{whale}
%设置字体主题;这个主题一般选择海洋动物来命名
%这里还可以选择别的颜色主题,如默认的和有特别目的的颜色主题whale,seahorse,dolphin

%\usefonttheme{professionalfonts}
%类似的还可以定义structurebold,structuresmallcapsserif,professionalfonts


% 控制 beamer 的风格,可以根据自己的爱好修改
%\usepackage{beamerthemesplit} %使用 split 风格
%\usepackage{beamerthemeshadow} %使用 shadow 风格
%\usepackage[width=2cm,dark,tab]{beamerthemesidebar}


% 设定英文字体
\usepackage{fontspec}
\usepackage{ listings}
\setmainfont{DejaVu Sans}
%\setmainfont{Times New Roman}
%\setsansfont{Arial}
%\setmonofont{Courier New}

% 设定中文字体
\usepackage[BoldFont,SlantFont,CJKchecksingle,CJKnumber]{xeCJK}
%\setCJKmainfont[BoldFont={Adobe Heiti Std},ItalicFont={Adobe Kaiti Std}]{Adobe Song Std}
\setCJKmainfont[BoldFont={WenQuanYi Micro Hei},ItalicFont={WenQuanYi Micro Hei}]{WenQuanYi Micro Hei}
\setCJKsansfont{WenQuanYi Micro Hei}
\setCJKmonofont{WenQuanYi Micro Hei}
\punctstyle{hangmobanjiao}

\defaultfontfeatures{Mapping=tex-text}
\usepackage{xunicode}
\usepackage{xltxtra}

\XeTeXlinebreaklocale "zh"
\XeTeXlinebreakskip = 0pt plus 1pt minus 0.1pt

\usepackage{setspace}
\usepackage{booktabs}
\usepackage{colortbl,xcolor}
\usepackage{hyperref}
%\hypersetup{xetex,bookmarksnumbered=true,bookmarksopen=true,pdfborder=1,breaklinks,colorlinks,linkcolor=blue,filecolor=black,urlcolor=cyan,citecolor=green}
\hypersetup{xetex,bookmarksnumbered=true,bookmarksopen=true,pdfborder=1,breaklinks,colorlinks,linkcolor=cyan,filecolor=black,urlcolor=blue,citecolor=green}

% 插入图片
\usepackage{graphicx}
% 指定存储图片的路径(当前目录下的figures文件夹)
\graphicspath{{figures/}}

% 可能用到的包
\usepackage{amsmath,amssymb}
\usepackage{multimedia}
\usepackage{multicol}

% 定义一些自选的模板,包括背景、图标、导航条和页脚等,修改要慎重
% 设置背景渐变由10%的红变成10%的结构颜色
%\beamertemplateshadingbackground{red!10}{structure!10}
%\beamertemplatesolidbackgroundcolor{white!90!blue}
% 使所有隐藏的文本完全透明、动态,而且动态的范围很小
\beamertemplatetransparentcovereddynamic
% 使itemize环境中变成小球,这是一种视觉效果
\beamertemplateballitem
% 为所有已编号的部分设置一个章节目录,并且编号显示成小球
\beamertemplatenumberedballsectiontoc
% 将每一页的要素的要素名设成加粗字体
\beamertemplateboldpartpage

% item逐步显示时,使已经出现的item、正在显示的item、将要出现的item呈现不同颜色
\def\hilite<#1>{
 \temporal<#1>{\color{gray}}{\color{blue}}
    {\color{blue!25}}
}

% 自定义彩色块状结构的颜色
\setbeamercolor{bluecyanbgcolor}{fg=blue,bg=cyan}
\setbeamercolor{redblackbgcolor}{fg=red,bg=black}
\setbeamercolor{greenwhitebgcolor}{fg=green,bg=white}
\setbeamercolor{blueblackbgcolor}{fg=blue,bg=black}
\setbeamercolor{yellowmagentabgcolor}{fg=yellow,bg=green}
\setbeamercolor{blackorangebgcolor}{fg=black,bg=orange}
\setbeamercolor{magentabgcolor}{fg=,bg=magenta}



% 在表格、图片等得标题中显示编号
\setbeamertemplate{caption}[numbered]

% 打开PDF后直接全屏
\hypersetup{pdfpagemode={FullScreen}}
