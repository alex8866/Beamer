\input{beamer_head}

\usepackage{bbding}% 手势 \HandRight \HandLeft %\FiveStar \FourStar \SixStar
\newcommand{\handr}{\textcolor{magenta}{\HandRight}} % 自定义\handr

% 黄老师模板里的内容
\iffalse
\setbeamertemplate{navigation symbols}{}
\DeclareMathOperator{\arccot}{arccot}
\def\hilite<#1>{%
\temporal<#1>{\color{blue!35}}{\color{magenta}}%
{\color{blue!75}}}
\def\hidark<#1>{%
\temporal<#1>{\color{black!35}}{\color{magenta}}%
{\color{black!75}}}

\usepackage{wasysym}
\graphicspath{{figures/}}         %% 图片路径. 本文的图片都放在这个文件夹里了.
\usepackage{mflogo}          % \MP 得 MetaPost 字样
\fi

\iffalse
% beamer所有技巧:
% 设置item中项目符号为大于号
\begin{itemize}
          \item[\(>\)] everydaylife product
          \item[\(>\)] topic from your research group
\end{itemize}

% Text color
\textcolor{orange}{WinEdt}
\textcolor{gray}{WinEdt}
\textcolor{brown}{WinEdt}
\textcolor{red}{WinEdt}
\textcolor{blue}{WinEdt}
\textcolor{green}{WinEdt}

% 取消导航
%\setbeamertemplate{navigation symbols}{}

% 定义颜色
%\definecolor{blendedred}{rgb}{0.7,0.2,0.2} % use structure theme to change
%\definecolor{beamer@blendedblue}{rgb}{0.2,0.2,0.7} % use structure theme to change

% 设定结构的前景色
%\setbeamercolor{structure}{fg=beamer@blendedred}

%\usefonttheme[stillsansseriflarge,stillsansserifsmall]{serif}

% 同一个位置动态显示两个block,用下面方法
\begin{frame}
     \only<1>{\begin{itemize}
     \item item1
     \item item2
     \end{itemize}}

     \only<2>{\begin{itemize}
     \item item3
     \item item4
     \end{itemize}}
\end{frame}

方法2:
\only<2>{
\begin{exampleblock}{问与答}
\alert{问:} 1
\alert{答:} 1
\end{exampleblock}}

\only<3->{
\begin{exampleblock}{问与答}
\alert{问:}2
\alert{答:}2
\end{exampleblock}}

% 字体颜色大小
{\huge \emph{\textcolor{blue!80}{Thank  ~you!}}}\\

% 设置返回button
\hfill \hyperlink{target4}{\beamergotobutton{返回}}

%字体背景
\colorbox{yellow!85}{This is a test}

% 设置手指标示:
\item[\handr] 不可目中无人

%url
\url{http://www.ctex.org/}

% block
\begin{block}{}
some text
\end{block}

% 以下命令中通过选择show或者hide控制相应节的显示
\tableofcontents[currentsection,currentsubsection,subsectionstyle=show/show/hide]
% 斜体字
\emph{\decmagenta{6:00-6:30} 起床 \& 洗漱}\\
\emph{\decmagenta{6:30-8:00} 看书}

% 时间
\date{\today}

% 取消/显示导航条
\setbeamertemplate{navigation symbols}{}

% 设定不显示导航栏
\setbeamertemplate{navigation symbols}{}

% 设定底栏只显示页码
%\setbeamertemplate{footline}[frame number]

% 以下命令中通过选择show或者hide控制相应节的显示
\tableofcontents[currentsection,currentsubsection,subsectionstyle=hide]

% 描述列表环境
\begin{description}
    \item [孔]飞,飞
    \item [丽]丽理
\end{description}

% beamer button
  \begin{itemize}
  \item Use \textbf{hypertarget} add hyper link.\\
    $\backslash hypertarget<\mbox{overlay-specification}>\{\mbox{target-name}\{\mbox{text}\}$
    \hyperlink{jumptosecond}{\beamergotobutton{Jump to second slide}}
    \hypertarget<2>{jumptosecond}{}
  \item \beamerbutton{beamerbutton}
  \item \beamerskipbutton{beamerskipbutton}
  \item \beamerreturnbutton{beamerreturnbutton}
  \end{itemize}

% switch action
\section{Graphics, Animations, sounds, and Slide Transitions }
\frame{
  \frametitle{\secname}
  \begin{itemize}
  \item Graphics \\
    \includegraphics<1>{pku-logo.pdf}
    \includegraphics<2>{pku-tower.pdf}
    \pgfuseimage<3>{logo}
  \item Animations \\
    \movie[externalviewer,label=mymovie,width=1in,height=0.8in,poster]{}{movie.avi}
%    \hyperlinkmovie[play]{mymovie}{Play}
  \item Sound
    \movie[externalviewer,autostart]{Here's some music}{turky.mp3}
  \end{itemize}
}

\fi
%\setbeamercolor{frametitle}{fg=green}
% 使用 \part,\section,\subsection,subsubsection 等命令组织文档结构
% 使用 \frame 命令制作幻灯片
\def\decred#1{{\color{red!75!black}#1}}
\def\decblue#1{{\color{blue!75!black}#1}}
\def\decmagenta#1{{\color{magenta!75!black}#1}}
\begin{document}

\logo{\includegraphics[height=0.09\textwidth]{lkong.png}}
\title[\textbf{\textcolor{red}{请遵守!}}(依心而变)]{\textcolor{red}{Life Rules}}
\author[智慧人生]{Lingfei Kong}
\institute[Running]{未雨绸缪}
\date{January 01, 2015}

% 定义目录页
\AtBeginPart{
  \frame{
    \frametitle{\partpage}
    \begin{multicols}{2}
% 如果目录过长,可以打开这个选项分两栏显示
      \tableofcontents
% 使用这个命令自动生成目录
    \end{multicols}
  }
}

%\AtBeginSsection[]
%{
%  \begin{frame}<beamer>
%    \frametitle{Agenda}
%	\setcounter{tocdepth}{2}
%    \tableofcontents[currentsection,currentsubsection]
%  \end{frame}
%}

% 在每个Section前都会加入的Frame
\iffalse
\AtBeginSection[]
{
  \begin{frame}<beamer>
    \frametitle{Agenda}
    \begin{multicols}{2}
	\setcounter{tocdepth}{1}
    \tableofcontents[currentsection]
    \end{multicols}
  \end{frame}
}
% 在每个Subsection前都会加入的Frame
\AtBeginSubsection[]
{
  \begin{frame}<beamer>
%\begin{frame}<handout:0>
% handout:0 表示只在手稿中出现
    \frametitle{Agenda}
    \begin{multicols}{2}
	\setcounter{tocdepth}{2}
    \tableofcontents[currentsection,currentsubsection]
    \end{multicols}
% 显示在目录中加亮的当前章节
  \end{frame}
}
\fi

\iffalse
\AtBeginSubsubsection[]
{
  \begin{frame}<beamer>
%\begin{frame}<handout:0>
% handout:0 表示只在手稿中出现
    \frametitle{Agenda}
    \begin{multicols}{2}
	\setcounter{tocdepth}{3}
    \tableofcontents[currentsection,currentsubsection]
    \end{multicols}
% 显示在目录中加亮的当前章节
  \end{frame}
}
\fi

%\theoremstyle{block}
%\newtheorem{EK}{Emacs Key}
%\theoremstyle{definition}
%\newtheorem{EC}{Emacs Configure}
%\theoremstyle{example}
%\newtheorem{IP}{Install Packages}

\defverbatim[colored]\etags{%
\begin{lstlisting}[frame=single, emph={find}, emphstyle={\color{blue}}]
find . -name "*.py" -print | etags -
\end{lstlisting}
}

% 设定不显示导航栏
\setbeamertemplate{navigation symbols}{}

% 设定底栏只显示页码
%\setbeamertemplate{footline}[frame number]

\begin{frame}
  \titlepage
\end{frame}

\begin{frame}[plain]
  \frametitle{Agenda}
  \begin{multicols}{2}
  \setcounter{tocdepth}{2}
  \tableofcontents
  \end{multicols}
\end{frame}

\section{生命不息,奋斗不息!}
\begin{frame}
%	\frametitle{Richard Stallman}
	\begin{figure}[http]
    \centering
    \includegraphics[width=14cm]{time.jpg}
%    \caption{想成功就遵守}
    \label{fig:time}
    \end{figure}
\end{frame}

\section{四字真言}
\begin{frame}{四字真言}
    \transdissolve
%    \frametitle{四字真言}
    \begin{columns}
    \column{5cm}
    \begin{itemize}
%        \transblindshorizontal<1>
        \item \textcolor{red}{健康生活}
        \item \textcolor{red}{谦虚谨慎}
        \item \textcolor{red}{勤俭节约}
        \item \textcolor{red}{宽厚守信}
    \end{itemize}
    \column{5cm}
    \begin{itemize}
        \item \textcolor{red}{未雨绸缪}
        \item \textcolor{red}{活跃思维}
        \item \textcolor{red}{珍时学习}
        \item \textcolor{red}{刚毅不懈}
    \end{itemize}
    \end{columns}
\end{frame}
\section{生活篇}
\subsection{生活规律}
\begin{frame}
	\frametitle{作息规律}
    \begin{beamercolorbox}[rounded=true,shadow=true,wd=12cm]{bluecyanbgcolor}
    早晨: \textcolor{red}{06:00} 起床\\
    \begin{itemize}
        \item \textcolor{red}{6:00-6:30} 起床 \& 洗漱
        \item \textcolor{red}{6:30-8:00} 看书
    \end{itemize}
    \end{beamercolorbox}
    \begin{beamercolorbox}[rounded=true,shadow=true,wd=12cm]{blackorangebgcolor}
    %\begin{beamerboxesrounded}[upper=block head,lower=block body,shadow=true]{}
    中午: \textcolor{red}{12:30-13:00} 午休\\
    %\end{beamerboxesrounded}
    \end{beamercolorbox}
    \begin{beamercolorbox}[rounded=true,shadow=true,wd=12cm]{redblackbgcolor}
    晚上: \textcolor{orange}{23:30} 睡觉\\
    \begin{itemize}
    \item 看书学习
    \item 思考 \& 处理问题
    \item \textcolor{orange}{21:30} 如有需要,则进入研究阶段
    \item \textcolor{orange}{23:00-23:30} 锻炼 ,比如:臂力,俯卧撑,仰卧起坐,手力,压腿等,然后洗漱。
    \item \textcolor{orange}{23:30} 给明天做计划,而后入睡思考问题
    \end{itemize}
    \end{beamercolorbox}
%    \begin{beamercolorbox}[rounded=true,shadow=true,wd=12cm]{redblackbgcolor}
%    \begin{beamerboxesrounded}[upper=block head,lower=block body,shadow=true]{}
\end{frame}

\begin{frame}
	\frametitle{作息规律}
	\begin{alertblock}{其他时间}
    \begin{itemize}
    \item \textcolor{green}{周六} 可07:00起,工作学习
    \item \textcolor{green}{周日} 可08:00起,上午:洗衣服,处理杂事。下午:休息/娱乐/处理事情
    \item 休息假期=\{1/2*假期\} (视情况而定)
    \item 集聚/旅游/其他 - 社交/思考问题/认真听讲,发散思维
    \item 走路 - 思考问题/认真观察
    \item 其他:enjoy it!
    \end{itemize}
    \end{alertblock}
\end{frame}

\subsection{兴趣爱好}
\begin{frame}
\frametitle{兴趣爱好}
\begin{itemize}
    \item 听音乐
    \item 爬山, 跑步
    \item 看书
\end{itemize}
\end{frame}

\subsection{生活习惯}
\begin{frame}
\frametitle{\textcolor{red}{生活习惯}}
\begin{itemize}
    \item \textcolor{red}{养成良好\&健康的生活习惯}
    \item 作息有规律
    \item 拒绝黄赌毒,确保身心健康
    \item 干净整洁
    \item 坐有相,站有姿,尽显大将风范
\end{itemize}
\end{frame}

\section{做人做事篇}
\subsection{做人}
\begin{frame}
\frametitle{做人}
\begin{itemize}
    \item 诚实守信, 雷厉风行
    \item 思路清晰,逻辑严密又不失幽默
    \item 放下面子,见什么人说什么话(厚黑学)
    \item 自信—任你东南西北风
    \item 沉着冷静心自怡, 成熟理智
\end{itemize}
\end{frame}

\subsection{做事}
\begin{frame}
\frametitle{做事}
\begin{itemize}
    \item 高效
    \item 有计划,有步骤
\end{itemize}
\end{frame}

\section{学习篇}
\begin{frame}{学习篇}
\begin{itemize}
    \item 学习内容: 书本知识 + 社会知识
    \item 珍惜一切时间来学习思考, 贵在坚持
    \item 活跃思维,发散思维,未雨绸缪
\end{itemize}
\end{frame}

\iffalse
\definecolor{bottomcolor}{rgb}{0.32,0.3,0.38}
\definecolor{middlecolor}{rgb}{0.08,0.08,0.16}
\setbeamertemplate{background canvas}[vertical shading]

\setbeamercolor{title}{fg=yellow,bg=gray}
\setbeamercolor{normal text}{fg=white,bg=black}

\setbeamercolor{title}{fg=yellow,bg=gray}
\setbeamerfont{title}{size=\LARGE}
\setbeamertemplate{section in toc}[sections numbered]
\setbeamercolor{section in toc}{fg=yellow!80!gray}
\fi

\setbeamertemplate{itemize items}{\color{cyan}$\clubsuit$}
%\setbeamertemplate{items}[ball]
%\setbeamertemplate{items}[square]
%\setbeamertemplate{itemize items}{\color{blue}$\bullet$}
%\setbeamertemplate{itemize items}{\color{blue}$\checkmark$}
%\setbeamertemplate{items}[triangle]
%\setbeamertemplate{items}[circle]


\section{其他Rules篇}
\begin{frame}{其他Rules篇}
\begin{itemize}
    \item 要想别人尊重你,那么你首先要学会尊重别人
    \item 说出的话必须经过脑子
\end{itemize}
\end{frame}

\iffalse
\begin{frame}[fragile]{抄录环境}
这是一段抄录代码:\verb!\frame{hello beamer}!。
\end{frame}
\fi

%\setbeamertemplate{itemize items}{\color{red}$\bullet$}
\setbeamertemplate{itemize items}{\color{red}$\checkmark$}
%\setbeamertemplate{items}[ball]
%\setbeamertemplate{items}[triangle]

\section{感悟篇}
\begin{frame}{感悟篇}
\begin{itemize}
    \item 要想别人尊重你,那么你首先要学会尊重别人
    \item 说出的话必须经过脑子
\end{itemize}
\end{frame}

\begin{frame}
\begin{itemize}
    \item 不想让别人知道的事情就一定不要说
\end{itemize}
\end{frame}

\setbeamertemplate{items}[ball]

\section{人生宝典}
\begin{frame}
\frametitle{人生宝典}
\begin{beamercolorbox}[rounded=true,shadow=true,wd=12cm]{redblackbgcolor}
    一个人不管有多聪明,多能干,背景条件有多好,如果不懂得如何去做人、做事,那么他最终的结局肯定是失败。做人做事是一门艺术,更是一门学问。很多人之所以一辈子都碌碌无为,那是因为他活了一辈子都没有弄明白该怎样去做人做事。
~\\
    \begin{itemize}
        \item 社会交往字诀
        \item 形象塑造字诀
        \item 自我提升字诀
        \item 人际互动字诀
        \item 解困渡厄字诀
        \item 不败人生字诀
    \end{itemize}
\end{beamercolorbox}
\end{frame}

%\setbeamertemplate{itemize items}{\color{blue}$\bullet$}
%\setbeamertemplate{itemize items}{\color{blue}$\checkmark$}
%\setbeamertemplate{items}[triangle]
\subsection{社会交往字诀}
\begin{frame}
\frametitle{社会交往字诀}
\begin{itemize}
    \item \textcolor{brown}{"谦"}字诀
    \item \textcolor{brown}{"淡"}字诀
    \item \textcolor{brown}{"俭"}字诀
    \item \textcolor{brown}{"自"}字诀
    \item \textcolor{brown}{"礼"}字诀
    \item \textcolor{brown}{"正"}字诀
\end{itemize}
\end{frame}

\setbeamertemplate{itemize items}{\color{red}$\bullet$}
%\setbeamertemplate{items}[ball]
\subsubsection{"谦"字诀}
\begin{frame}
\frametitle{"谦"字诀}
\noindent\handr~~处世唯"谦"字了得,若一味狂妄自负、骄傲自大,只会失去处世的根本,落得个孤苦伶仃、千夫所指的骂名下场。
~\\
~\\
\begin{itemize}
    \item 不可目中无人
    \item 得意不要忘形
    \item 有本事不必自夸
    \item 请教不择人
\end{itemize}
\end{frame}

\subsubsection{"淡"字诀}
\begin{frame}
\frametitle{"淡"字诀}
\noindent\handr~~为人处世,交朋待友,对势利纷华,似乎不必太过于苛求,当以"淡"字当头。看淡些,看开些,人生也就豁然开朗,有滋有味了。 正如"平平淡淡才是真"。
~\\
~\\
\begin{itemize}
    \item 君子之交淡如水
    \item 淡看人生,善待生命
    \item 淡泊明志,莫为名利遮望眼
    \item 减少心欲,满足心灵
\end{itemize}
\end{frame}

\subsubsection{"俭"字诀}
\begin{frame}
\frametitle{"俭"字诀}
\noindent\handr~~不懂得"俭"字的人,不知道如何成功,任何成功的事业都在于点滴的积累;不懂得"俭"字的人,只会丧失成功,过分的骄奢多败人品质。"俭以养德",为人做事之良训。
~\\
~\\
\begin{itemize}
    \item 从节省生活费开始
    \item "穷大方"不可取
    \item 谨防变态的节俭:吝啬
    \item 欲路勿染,俭以养德
\end{itemize}
\end{frame}

\subsubsection{"自"字诀}
\begin{frame}
\frametitle{"自"字诀}
\noindent\handr~~做一个有个性的人,给自己一点自信!成功的道路靠自己闯,美好的前途来自于自强自立,不屈服于任何权威,用自我的努力找到属于你的自尊。男儿立世,自己拍板!
~\\
~\\
\begin{itemize}
    \item 自强自立,与成功有约
    \item 独品人生百态
    \item 用自我来挑战权威
    \item 自信—任你东南西北风
\end{itemize}
\end{frame}

\subsubsection{"礼"字诀}
\begin{frame}
\frametitle{"礼"字诀}
\noindent\handr~~生在礼仪之邦,做一个彬彬有礼之人。有礼之人会做人,有人缘,多朋友。有礼之人会做事,注重形象,有教养,不树敌,成功路上事事顺。
~\\
~\\
\begin{itemize}
    \item 以礼待人
    \item 彬彬有礼,礼多人不怪
    \item 注重礼仪着装,给人良好印象
\end{itemize}
\end{frame}

\subsubsection{"正"字诀}
\begin{frame}
\frametitle{"正"字诀}
\noindent\handr~~做一个正直的人,做一个人格健全完善的人,受人崇敬。做一个自私的人,做欺心的事,疾贤防能,与成功无缘。
~\\
~\\
\begin{itemize}
    \item 己所不欲,勿施于人
    \item 嫉妒乃方正之人之大忌
    \item 不做欺心事,本身是一种愉悦
\end{itemize}
\end{frame}

%\setbeamertemplate{itemize items}{\color{red}$\bullet$}
\setbeamertemplate{items}[ball]

\subsection{形象塑造字诀}
\begin{frame}
\frametitle{形象塑造字诀}
\begin{itemize}
    \item \textcolor{brown}{"志"}字诀
    \item \textcolor{brown}{"时"}字诀
    \item \textcolor{brown}{"勤"}字诀
    \item \textcolor{brown}{"实"}字诀
    \item \textcolor{brown}{"专"}字诀
    \item \textcolor{brown}{"慎"}字诀
\end{itemize}
\end{frame}

\setbeamertemplate{itemize items}{\color{red}$\bullet$}
%\setbeamertemplate{items}[ball]

\subsubsection{"志"字诀}
\begin{frame}
\frametitle{"志"字诀}
\noindent\handr~~给自己一根足够长的杠杆,希望转动地球。给自己的人生立个志愿,树个目标,树个偶像,脚踏实地,成功的意识需要培养,先立志,再与成功约会。
~\\
~\\
\begin{itemize}
    \item 度德量力,以志立身
    \item 先立志,有志就有希望
    \item 培养成功意识:立志为王
    \item 树立偶像,改变自己
\end{itemize}
\end{frame}

\subsubsection{"时"字诀}
\begin{frame}
\frametitle{"时"字诀}
\noindent\handr~~做人要惜时,做事要守时。塑造自己的形象,现代人离不开时间观念。合理安排自己的时间,有效利用自己的时间,守时、惜时、不拖延。 切记:时间就是金钱。
~\\
~\\
\begin{itemize}
    \item 一秒值万金
    \item 别漠视业余时间
    \item 盗窃他人时间,等于谋财害命
    \item 按重要性办事,更能有效利用时间
\end{itemize}
\end{frame}

\subsubsection{"勤"字诀}
\begin{frame}
\frametitle{"勤"字诀}
\noindent\handr~~多一些努力,便多一些成功的机会。无数事实证明:成功的最短途径是勤奋。不要光耍嘴皮子,不要好逸恶劳,勤字当头,苍天不负有心人,天道酬勤!
~\\
~\\
\begin{itemize}
    \item 成功的最短途径:勤奋
    \item 多一些努力,多一些机会
    \item 勤于行动,胜于勤说
\end{itemize}
\end{frame}

\subsubsection{"实"字诀}
\begin{frame}
\frametitle{"实"字诀}
\noindent\handr~~踏踏实实做人,实实在在办事。任何一个双手插在口袋里的人,都爬不上成功的梯子。给人留下一个实在的形象,给自己的成功增添一份夯实的基础,从实际出发,对自己负责。
~\\
~\\
\begin{itemize}
    \item 敬业,实干家的成功保障
    \item 把每一份工作都做好
    \item 双手插在口袋里的人,爬不上成功的梯子
\end{itemize}
\end{frame}

\subsubsection{"专"字诀}
\begin{frame}
\frametitle{"专"字诀}
\noindent\handr~~有专才有恒,有恒才有我。 你生活在一个知识大爆炸的时代,如果你是一个天才,不专心就成了你的不幸;如果你资质平凡,请不要悲观,只要你下定决心一辈子做好一件事,你就能成功。年轻人,千万别给人留下一个朝三暮四的形象。
~\\
~\\
\begin{itemize}
    \item 把所有的鸡蛋放入一个篮子
    \item 多才多艺,莫如练就"独门暗器"
    \item 专一,让劣势变成优势
\end{itemize}
\end{frame}

\subsubsection{"慎"字诀}
\begin{frame}
\frametitle{"慎"字诀}
\noindent\handr~~人生漫长,又短暂,关键的就几步。人性丛林,职场事业,利益多多、诱惑多多。老成不怕多,凡事应多三思,不怕一万,就怕万一。一旦伸错手,入错行,做错事,于名誉,于事业,于形象皆有不救之危。 "慎"之!
~\\
~\\
\begin{itemize}
    \item 千万别入错行
    \item 想好了你再"跳"
    \item 不要草率行事
\end{itemize}
\end{frame}

%\setbeamertemplate{itemize items}{\color{red}$\bullet$}
\setbeamertemplate{items}[ball]

\subsection{自我提升字诀}
\begin{frame}
\frametitle{自我提升字诀}
\begin{itemize}
    \item \textcolor{brown}{"硬"}字诀
    \item \textcolor{brown}{"小"}字诀
    \item \textcolor{brown}{"锐"}字诀
    \item \textcolor{brown}{"创"}字诀
    \item \textcolor{brown}{"通"}字诀
    \item \textcolor{brown}{"言"}字诀
\end{itemize}
\end{frame}

\setbeamertemplate{itemize items}{\color{red}$\bullet$}
%\setbeamertemplate{items}[ball]

\subsubsection{"硬"字诀}
\begin{frame}
\frametitle{"硬"字诀}
\noindent\handr~~做人难,做事难,面对千难万阻,要提升自我,不来点"硬"的怎么行?如果事有勉强,应该敢于说"不";如果是正当利益,则应当仁不让;甚至,有时还得来点霸王硬上弓,要有"脸皮厚"的时候,也要有"头皮硬"的时候。
~\\
~\\
\begin{itemize}
    \item 拒绝是一门艺术
    \item 该我的,就不要客气
    \item 怒发冲冠之功
    \item 厚脸皮做人,硬头皮做事
\end{itemize}
\end{frame}

\subsubsection{"小"字诀}
\begin{frame}
\frametitle{"小"字诀}
\noindent\handr~~一家海鲜连锁餐厅的老板很可能当初是水产市场练滩儿的,而一家皮鞋连锁店的老板当初可能是擦鞋的。欲做大事,赚大钱,必先做小事,赚小钱,放下架子,舍得小利。从细微处入手,先扫一屋,再扫天下!
~\\
~\\
\begin{itemize}
    \item 一屋不扫,何以扫天下
    \item 先做小事,赚小钱
    \item 一枚钉子改变一个人的一生
\end{itemize}
\end{frame}

\subsubsection{"锐"字诀}
\begin{frame}
\frametitle{"锐"字诀}
\noindent\handr~~小小麻雀,飞飞跳跳、争分夺秒,不停地寻觅食物。人生亦如此,面对残酷竞争,惟有锐意进取,做一个好先锋,把下一个进球当目标,敢于冒险,敢于闯荡,守株待兔的事情毕竟很渺茫。
~\\
~\\
\begin{itemize}
    \item 不以现有成就为满足
    \item 锐意追求,绝不退缩
    \item 锐气不可抛,成功是迟早
\end{itemize}
\end{frame}

\subsubsection{"创"字诀}
\begin{frame}
\frametitle{"创"字诀}
\noindent\handr~~提升自我,就要有胆有识去超越自我。何谓超越?超越就是吃螃蟹,就是创新。同时创新就意味着冒险,所谓富贵险中求。想人家想不到的,做别人不敢做的,敢为天下先,在于思维的转换。
~\\
~\\
\begin{itemize}
    \item 敢为天下先
    \item 打破规则的创意
    \item 人弃我取也能创奇迹
    \item 逆向思维的攻守之道
\end{itemize}
\end{frame}

\subsubsection{"通"字诀}
\begin{frame}
\frametitle{"通"字诀}
\noindent\handr~~穷则思变,变则通。识时务者为俊杰,通机变者为英豪。通往成功的道路不是一条,又何必在一棵树上吊死呢?抓住成功的关键,东方不亮西方亮,不管它是黑猫白猫,重要的是它能否逮"耗子"。
~\\
~\\
\begin{itemize}
    \item 巧妙地以变应变
    \item 条条大道通罗马
    \item 成功在于通,有通才有赢
\end{itemize}
\end{frame}

\subsubsection{"言"字诀}
\begin{frame}
\frametitle{"言"字诀}
\noindent\handr~~把赞扬送给别人,就像把食物施舍给饥饿的乞丐一样。古往今来,不知有多少人,凭着三寸不烂之舌,改变了自己平凡的命运。说话幽默,找共同语言……一个"言"字,一生受用。
~\\
~\\
\begin{itemize}
    \item 投其所好找话题
    \item 恭维是最好的"润滑剂"
    \item 成功人生,幽默机智
    \item "流行语"为你添姿着色
\end{itemize}
\end{frame}

%\setbeamertemplate{itemize items}{\color{red}$\bullet$}
\setbeamertemplate{items}[ball]

\subsection{人际互动字诀}
\begin{frame}
\frametitle{人际互动字诀}
\begin{itemize}
    \item \textcolor{brown}{"宽"}字诀
    \item \textcolor{brown}{"和"}字诀
    \item \textcolor{brown}{"信"}字诀
    \item \textcolor{brown}{"帮"}字诀
    \item \textcolor{brown}{"敬"}字诀
    \item \textcolor{brown}{"交"}字诀
\end{itemize}
\end{frame}

\setbeamertemplate{itemize items}{\color{red}$\bullet$}
%\setbeamertemplate{items}[ball]

\subsubsection{"宽"字诀}
\begin{frame}
\frametitle{"宽"字诀}
\noindent\handr~~人际互动,应着眼于未来,不念旧恶。原谅别人,是对待自己的最好方式——为你的仇敌而怒火中烧,烧伤的是你自己。做人做事,心胸不可太狭隘。海纳百川,靠一棵宽容的心!
~\\
~\\
\begin{itemize}
    \item 宽恕你的敌人
    \item 宽容做人,宽容成事
    \item 乐于忘记,不念旧恶
\end{itemize}
\end{frame}

\subsubsection{"和"字诀}
\begin{frame}
\frametitle{"和"字诀}
\noindent\handr~~在人海中,如果我们不想孤立,那么就学会如何与人相处吧!林子大了,什么鸟都有,不要求你喜欢所有的人,但同时世上也没有什么最牛的人。 和为贵嘛,就要互相留台阶,大家给面子。
~\\
~\\
\begin{itemize}
    \item 为他人着想,为自己铺路
    \item 你给别人留面子,别人给你做好事
    \item 夫妻之道,亦和亦智
\end{itemize}
\end{frame}

\subsubsection{"信"字诀}
\begin{frame}
\frametitle{"信"字诀}
\noindent\handr~~有多少人信任你,你就拥有多少次成功的机会,"信"是什么东西?信是一种人格的力量,是超越金钱的友情,是了解、是欣赏、是覆水,具有不可逆转性。所以,言必行,行必果,能帮的忙则帮,但不可轻易许诺!
~\\
~\\
\begin{itemize}
    \item 能帮则帮,不轻易许诺
    \item 言而有信,做人讲原则
    \item 做事先做人,做人先取信
    \item 信誉基石,生死友情
\end{itemize}
\end{frame}

\subsubsection{"帮"字诀}
\begin{frame}
\frametitle{"帮"字诀}
\noindent\handr~~"好风凭借力,送我上青天"。人际交往,互利互惠。帮助别人,就是在为自己的人情信用卡储蓄,特别是在人患难之际施于援手,救落难英雄于困顿。真心助人,其回报不言而喻。
~\\
~\\
\begin{itemize}
    \item 助人发财,自己沾光
    \item 好风凭借力,借梯能登天
    \item 掌握时机,拉人一把
\end{itemize}
\end{frame}

\subsubsection{"敬"字诀}
\begin{frame}
\frametitle{"敬"字诀}
\noindent\handr~~人要面子树要皮。人存在于社会上,要扮演各种各样角色,特别是在互相的交往中,需要一定的尊严来支撑,这是人性的弱点。明白了这点,才能体会到"敬"字的必要性。
~\\
~\\
\begin{itemize}
    \item 为尊者讳,为上司讳
    \item 在失意者面前不谈你的得意
    \item 尊敬对方的"闪光点"
\end{itemize}
\end{frame}

\subsubsection{"交"字诀}
\begin{frame}
\frametitle{"交"字诀}
\noindent\handr~~人情冷暖、世态炎凉,平常朋友平常过。交朋接友,不可急功近利,友情投资,宜走长线,拜拜冷庙,烧烧冷灶,平时多烧香,哪怕是只言片语的问候,亦是交友之道。
~\\
~\\
\begin{itemize}
    \item 闲时多烧香,急时有人帮
    \item 友情投资,宜走长线
    \item 拜冷庙,烧冷灶,交落难英雄
\end{itemize}
\end{frame}

%\setbeamertemplate{itemize items}{\color{red}$\bullet$}
\setbeamertemplate{items}[ball]

\subsection{解困渡厄字诀}
\begin{frame}
\frametitle{解困渡厄字诀}
\begin{itemize}
    \item \textcolor{brown}{"坚"}字诀
    \item \textcolor{brown}{"谋"}字诀
    \item \textcolor{brown}{"屈"}字诀
    \item \textcolor{brown}{"静"}字诀
    \item \textcolor{brown}{"乐"}字诀
    \item \textcolor{brown}{"靠"}字诀
\end{itemize}
\end{frame}

\setbeamertemplate{itemize items}{\color{red}$\bullet$}
%\setbeamertemplate{items}[ball]

\subsubsection{"坚"字诀}
\begin{frame}
\frametitle{"坚"字诀}
\noindent\handr~~面对挫折与困难,铭记丘吉尔的名言:"永远,永远,永远不要放弃!"其实世界上并没有什么幸运的事,就是有,也是坚持的结果。为了最后的胜利,应以坚毅不拔之志,面对种种暂时之屈辱,执着追求,不到黄河心不死!
~\\
~\\
\begin{itemize}
    \item 厚积薄发,耐得寂寞
    \item 谁笑到最后,谁笑得最甜
    \item 执着追求,永不放弃
    \item 不到黄河心不死
\end{itemize}
\end{frame}

\subsubsection{"谋"字诀}
\begin{frame}
\frametitle{"谋"字诀}
\noindent\handr~~做人有困惑,做事有困境,面对"山重水复"之关卡,光有坚强的毅志不行,硬闯也不行。解决难题靠的是脑袋,脑袋产生思考,让思考发威,在出人意料之处轻松解决问题。
~\\
~\\
\begin{itemize}
    \item 巧妇难为无米之炊
    \item 从"山重水复"到"柳暗花明"
    \item 思考的威力
\end{itemize}
\end{frame}

\subsubsection{"屈"字诀}
\begin{frame}
\frametitle{"屈"字诀}
\noindent\handr~~要摆脱人与事的困境,就难免要求人,求人就难免要低三下四,但着眼于未来的成功,即使像蟑螂一样的生活也应在所不惜,风水毕竟轮流转。放下架子,该屈就屈,能屈能伸,以屈为伸方为英雄!
~\\
~\\
\begin{itemize}
    \item 像蟑螂一样生活
    \item 放下身段,前方是大道
    \item 你敬我一尺,我敬你一丈
    \item 低人一级"屈"不死人
\end{itemize}
\end{frame}

\subsubsection{"静"字诀}
\begin{frame}
\frametitle{"静"字诀}
\noindent\handr~~"不在沉默中爆发,就在沉默中灭亡!"凡遇大事需静气,平心静气是一种境界,一种气度,一种修养。冷静之中的决定往往是摆脱困境的最佳方案,同时冷静也是一种智慧,以静待变,乱中取胜!
~\\
~\\
\begin{itemize}
    \item 把冷板凳坐成经理椅
    \item 心宁智生,智生事成
    \item 沉着冷静心自怡
    \item 沉得住气方为人杰
\end{itemize}
\end{frame}

\subsubsection{"乐"字诀}
\begin{frame}
\frametitle{"乐"字诀}
\noindent\handr~~世上没有绝对幸福的人,只有不肯快乐的心。人生苦短,与其事事张弓拔弩,不如"幽它一默"。记住,成功是从微笑开始的,人生不如意事常八九,乐观点,自己营造快乐,学会轻松解决难题。
~\\
~\\
\begin{itemize}
    \item 成功从微笑开始
    \item 学会营造快乐
    \item 学会轻松愉快地解决难题
    \item 世上没有绝对幸福的人,只有不肯快乐的心
\end{itemize}
\end{frame}

\subsubsection{"靠"字诀}
\begin{frame}
\frametitle{"靠"字诀}
\noindent\handr~~人生不等不靠,没错,天上不会掉馅饼,守株待兔饿死人,但一点不靠也不行,亲戚朋友、同学、老乡,这是一种"人力资源",谁人没个三灾六难,能靠则靠,靠不上创造条件也要靠!
~\\
~\\
\begin{itemize}
    \item 让朋友成为你的靠山
    \item 出门落难靠老乡
    \item 亲戚亲戚,越走越亲
    \item 恰同学少年,该靠靠一把
\end{itemize}
\end{frame}

%\setbeamertemplate{itemize items}{\color{red}$\bullet$}
\setbeamertemplate{items}[ball]

\subsection{不败人生字诀}
\begin{frame}
\frametitle{不败人生字诀}
\begin{itemize}
    \item \textcolor{brown}{"愚"}字诀
    \item \textcolor{brown}{"忍"}字诀
    \item \textcolor{brown}{"退"}字诀
    \item \textcolor{brown}{"圆"}字诀
    \item \textcolor{brown}{"危"}字诀
    \item \textcolor{brown}{"装"}字诀
\end{itemize}
\end{frame}

\setbeamertemplate{itemize items}{\color{red}$\bullet$}
%\setbeamertemplate{items}[ball]

\subsubsection{"愚"字诀}
\begin{frame}
\frametitle{"愚"字诀}
\noindent\handr~~学学猫头鹰,睁一只眼,闭一只眼。你说我糊涂,其实我不傻!只是世事多变幻,创业难,败家快,人说水至清则无鱼,人至察则无徒。其实是,明哲保身,大智者往往大愚,聪明者多,能过"愚"字关鲜矣!
~\\
~\\
\begin{itemize}
    \item 糊涂人聪明一世
    \item 不要以为自己比别人聪明
    \item 处事不要太认真
    \item 谁是英雄?
\end{itemize}
\end{frame}

\subsubsection{"忍"字诀}
\begin{frame}
\frametitle{"忍"字诀}
\noindent\handr~~真的英雄,何必气短,善始善终,方为不败!忍能保身,忍能成事,忍是大智,大勇,更是大福!忍是厚,忍是黑,忍小人,忍豪强,忍天下难忍之事,不做性情中人,成常人难成之事。
~\\
~\\
\begin{itemize}
    \item 忍是大智大勇大福
    \item 不做性情中人
    \item 不败人生,忍者无敌
\end{itemize}
\end{frame}

\subsubsection{"退"字诀}
\begin{frame}
\frametitle{"退"字诀}
\noindent\handr~~久历江湖,练达人情之人都守一个"退"字。退是一种谋略,退是一种交换,更是一种维系生存的手段。哲人说的好,"不要把痰吐在井里,哪天你口渴的时侯,也要来井边喝水的。"
~\\
~\\
\begin{itemize}
    \item 用心计较般般错,退步思量事事顺
    \item 拒绝妥协,就是拒绝成功
    \item 惹不起,躲得起
\end{itemize}
\end{frame}

\subsubsection{"圆"字诀}
\begin{frame}
\frametitle{"圆"字诀}
\noindent\handr~~方圆做人,八面玲珑;圆满做事,事事顺心。人心叵测,凡事最好留一手,有闲时,可研究一下"模糊哲学",人生这套马车,如若安上方方正正的轮子,你没听说过,我也没听说过,寸步难行嘛!
~\\
~\\
\begin{itemize}
    \item 方圆做人,圆满做事
    \item 做老二,不要做老大
    \item 人情练达即文章,处世圆通慎言语
\end{itemize}
\end{frame}

\subsubsection{"危"字诀}
\begin{frame}
\frametitle{"危"字诀}
\noindent\handr~~"豪华尽出成功后,逸乐安知与祸双?"历史教训如此,平头百姓亦如此。居家过日,工作职场等都逃不过一个"危"字,人无远虑,必有近忧。
~\\
~\\
\begin{itemize}
    \item 远虑在先,近处无危
    \item 郭子仪屏退侍女免祸患
    \item 上山下乡当农民——范蠡富贵终身
\end{itemize}
\end{frame}

\subsubsection{"装"字诀}
\begin{frame}
\frametitle{"装"字诀}
\noindent\handr~~人生在世一台戏,你方唱罢我上场,不管你会不会演,就看你会不会装。充英雄容易,扮弱者难。俗话说得好,枪打出头鸟,不怕贼偷就怕贼惦记着,当你还不具备实力时,请把你过剩的才华藏起来!
~\\
~\\
\begin{itemize}
    \item 故意示弱有好处
    \item 用"拟态"和"保护色"
    \item 成功需要诈死与佯败
\end{itemize}
\end{frame}

\section{祝你成功}
\begin{frame}{祝你成功!}
%	\frametitle{Richard Stallman}
	\begin{figure}[http]
    \centering
    \includegraphics[width=10cm]{run_time.jpg}
%    \caption{想成功就遵守}
    \label{fig:run_time}
    \end{figure}
\end{frame}

% Beamer 技巧篇
\iffalse
\section{Beamer技巧篇}

% 下面的代码修改了目录页中节标题的模板和颜色:
\setbeamertemplate{section in toc}[sections numbered]
\setbeamercolor{section in toc}{fg=yellow!80!gray}

\begin{frame}
\tableofcontents[hideallsubsections]
%\tableofcontents[currentsection,currentsubsection,subsectionstyle=hide]
\tableofcontents[currentsubsection,subsectionstyle=hide]
\end{frame}

\begin{frame}{无序列表样式}
\setbeamertemplate{itemize items}[default]
    \begin{itemize}
        \item default
    \end{itemize}

\setbeamertemplate{itemize items}[triangle]
    \begin{itemize}
        \item triangle
    \end{itemize}

\setbeamertemplate{itemize items}[circle]
    \begin{itemize}
        \item circle
    \end{itemize}

\setbeamertemplate{itemize items}[square]
    \begin{itemize}
        \item square
    \end{itemize}

\setbeamertemplate{itemize items}[ball]
    \begin{itemize}
        \item ball
    \end{itemize}

\end{frame}

\begin{frame}{无序列表}{这是一个副标题}
    \begin{itemize}
        \hilite <1> \item Beamer颜色参考:\textcolor{black}black,\textcolor{blue}{blue},\textcolor{red}{red},\textcolor{magenta}{magenta},\textcolor{green}{green},\textcolor{cyan}{cyan},\textcolor{yellow}{yellow},\textcolor{white}{white}
        \hilite <2-3> \item Beamer 演示中全部可以使用的字号如下:8pt、9pt、10pt、11pt、12pt、14pt、17pt、20pt,默认为 11pt。建议在较大的场合演示时使用大号的字体,例如:\\documentclass[14pt]\{beamer\}
        \hilite <3> \item 无序列表三
        \hilite <4-> \item 无序列表四
        \hilite <5> \item 无序列表五
        \hilite <-6> \item 无序列表六
        \hilite <7> \item 无序列表七
        \hilite <8> \item 无序列表八
        \hilite <9> \item 无序列表九
    \end{itemize}
\end{frame}

\begin{frame}{有序列表样式}
\setbeamertemplate{enumerate items}[default]
    \begin{enumerate}[<+-|alert@+>]
        \item default
    \end{enumerate}

\setbeamertemplate{enumerate items}[circle]

    \begin{enumerate}[<+-|alert@+>]
        \item circle
    \end{enumerate}

\setbeamertemplate{enumerate items}[square]

    \begin{enumerate}[<+-|alert@+>]
        \item square
    \end{enumerate}

\setbeamertemplate{enumerate items}[ball]

    \begin{enumerate}[<+-|alert@+>]
        \item ball
    \end{enumerate}
\end{frame}

\begin{frame}{有序列表}
    \begin{enumerate}[<+-|alert@+>]
        \item 有序列表一
        \item 有序列表二
        \item 有序列表三
        \item 有序列表四
        \item 有序列表五
        \item 有序列表六
        \item 有序列表七
        \item 有序列表八
        \item 有序列表九
    \end{enumerate}
\end{frame}

\begin{frame}{原地替换}
     \only<1>{\begin{itemize}
     \item item1
     \item item2
     \end{itemize}}

     \only<2>{\begin{itemize}
     \item item3
     \item item4
     \end{itemize}}
\end{frame}

\begin{frame}[plain]
        \begin{spacing}{1.5}
        \begin{center}
        \Huge{\textbf{Thanks for your attention!}}

        \Huge{\textit{Any questions?}}
\end{center}
\end{spacing}
\end{frame}

\begin{frame}{图片}
    \begin{figure}[htbp]
    \centering
    \includegraphics[width=8cm]{power.png}
    \caption{Powered by}
    \label{fig:power}
    \end{figure}
\end{frame}

\begin{frame}{表格}
    \begin{table}
    \centering
    \caption{一个表格示例}
    \rowcolors[]{1}{blue!20}{blue!10}
    \begin{tabular}{|c|c|c|c|}
    \hline
    \rowcolor{blue!50}学号&姓名&年龄&成绩\\
    \hline
    001 & 丁一 & 25 & 91\\
    002 & 刘二 & 24 & 89\\
    003 & 张三 & 25 & 90\\
    004 & 李四 & 24 & 95\\
    005 & 王五 & 25 & 93\\
    006 & 赵六 & 26 & 88\\
    007 & 钱七 & 25 & 86\\
    008 & 王八 & 24 & 89\\
    009 & 孙九 & 25 & 90\\
    \hline
    \end{tabular}
    \end{table}
\end{frame}

\begin{frame}{分栏}
        \begin{columns}
        \column{3cm}
        这是第一栏的文字;栏宽3cm。
        \column{5cm}
        这是第二栏的文字;栏宽5cm。
        \end{columns}
\end{frame}

\begin{frame}{方程}
        \begin{equation}
        E=mc^2
        \label{emc}
        \end{equation}
\end{frame}

\begin{frame}{描述列表环境}
\begin{description}
    \item[红色] 热情、活泼、温暖、幸福
    \item[绿色] 新鲜、平静、安逸、柔和
    \item[蓝色] 深远、永恒、沉静、寒冷
\end{description}
\end{frame}

%下面的代码修改了区块环境的样式:
\setbeamertemplate{blocks}[rounded][shadow=true]
\setbeamercolor{block title}{fg=yellow,bg=gray!50!black}
\setbeamercolor{block body}{bg=gray}

\begin{frame}{块}
    \begin{block}{重要例子}
    2012年12月21日是世界末日。
    \end{block}

    \begin{alertblock}{重要提醒}
    2012年12月21日是世界末日。
    \end{alertblock}

    \begin{exampleblock}{重要例子}
    2012年12月21日是世界末日。
    \end{exampleblock}

    \begin{theorem}
    微积分基本公式:$\int_a^b f(x)\mathrm{d}x=F(b)-F(a)$。
    \end{theorem}

    \begin{proof}
    令 $g(x)=e^x-x-1$。则当 $x>1$ 时, 有 $g'(x)=e^x-1>0$,
    因此 $g(x)>g(1)=0$。即有 $x>1$ 时 $e^x>1+x$。
    \end{proof}
\end{frame}

%下面的代码修改了正文字体的样式:
\setbeamercolor{normal text}{fg=red,bg=gray}
\setbeamerfont{frametitle}{size=\large}
\setbeamercolor{frametitle}{fg=yellow!70!gray}

\begin{frame}{修改正文字体}
~\\
~\\
~\\
~\\
This is a test
\end{frame}

\begin{frame}{Beamer功能}
\begin{itemize}
    \item 功能强大,各种侧栏、顶栏、底栏,导航栏一应俱全。
    \item 定制灵活,可以单独改变任何元素的结构,颜色和字体。
    \item 效果多样,支持各种各样的过渡效果并可以精确控制。
    \item 使用方便,可以用 latex,pdflatex 及 xelatex 编译。
\end{itemize}
\end{frame}

\begin{frame}{样式定制}
Beamer 的每个演示主题实际上都是由外部主题、内部主题、颜色主题和字体主题这四种细分主题组合而成的。如果要对演示主题作更加细致地选择,可以按照下面这四种细分主题自由组合:

\begin{itemize}
    \item 外部主题,用 \\usebeameroutertheme 命令;
    \item 内部主题,用 \\usebeamerinnertheme 命令;
    \item 颜色主题,用 \\usebeamercolortheme 命令;
    \item 字体主题,用 \\usebeamerfonttheme 命令。
    \item 参考网站: \url{http://deic.uab.es/~iblanes/beamer_gallery/} and \url{http://www.hartwork.org/beamer-theme-matrix/}

\end{itemize}
\end{frame}

\begin{frame}{外部主题}
外部主题设定演示文稿是否有顶栏、底栏和侧栏,以及它们的结构,可以用 \\useoutertheme\{主题名\} 来选择,其中主题名有如下这些选择:

\begin{columns}
    \column{3cm}
    \begin{itemize}
        \item default
        \item infolines
        \item miniframes
        \item sidebar
        \item smoothbars
    \end{itemize}

    \column{5cm}
    \begin{itemize}
        \item split
        \item shadow
        \item tree
        \item smoothtree
    \end{itemize}
\end{columns}
\end{frame}

\begin{frame}{内部主题}
内部主题设定演示文稿正文内容(例如标题、列表、定理等)的样式,可以用 \\useinnertheme\{主题名\} 来选择,其中主题名有如下这些选择:
\begin{itemize}
    \item default
    \item circles
    \item rectangles
    \item rounded
\end{itemize}
\end{frame}

\begin{frame}{颜色主题}
颜色主题设定演示文稿的各部分各结构各元素的配色,可以用\\usecolortheme\{主题名\} 来选择,其中主题名有这些选择:

\begin{description}
    \item [基本颜色] default、sidebartab、structure;
    \item [完整颜色] albatross(信天翁)、beaver(海狸)、beetle(甲壳虫)、crane(鹤)、dove(鸽子)、fly(苍蝇)、seagull(海鸥)、wolverine(狼獾);
    \item [内部颜色] lily(百合)、orchid(兰花)、rose(玫瑰);
    \item [外部颜色] dolphin(海豚)、seahorse(海马)、whale(鲸鱼)。
\end{description}
\end{frame}

\begin{frame}{字体主题}
字体主题设定演示文稿的字体,可以用 \\usefonttheme\{主题名\} 命令来选择,其中主题名有如下这些选择:
\begin{itemize}
    \item default
    \item serif
    \item structurebold
    \item structureitalicserif
    \item structuresmallcapsserif
\end{itemize}
\end{frame}

\frame[<+->]{
  \frametitle{Slide Transitions}
  \begin{enumerate}
  \item Horizontal blinds
    \transblindshorizontal<1>
  \item Vertical blinds
    \transblindsvertical<2>
  \item Moving to the center from all four sides
    \transboxin<3>
  \item Moving from the center to four sides
    \transboxout<4>
  \item Dissolve
    \transdissolve<5>
  \item Glitter
    \transglitter<5>
  \item Split verticalin
    \transsplitverticalin<7>
  \item Split verticalout
    \transsplitverticalout<8>
  \item wipe
    \transwipe<9>
  \item transduration
    \transduration<10>{1}
  \end{enumerate}
}

\fi

\end{document}
